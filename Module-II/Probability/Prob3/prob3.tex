\documentclass[journal,12pt,twocolumn]{IEEEtran}
\usepackage{graphicx}
\graphicspath{{./figs/}}{}
\usepackage{amsmath,amssymb,amsfonts,amsthm}
\newcommand{\myvec}[1]{\ensuremath{\begin{pmatrix}#1\end{pmatrix}}}
\usepackage{listings}
\usepackage{watermark}
\usepackage{titlesec}
\let\vec\mathbf

\titlespacing{\subsection}{0pt}{\parskip}{-3pt}
\titlespacing{\subsubsection}{0pt}{\parskip}{-\parskip}
\titlespacing{\paragraph}{0pt}{\parskip}{\parskip}
\newcommand{\figuremacro}[5]{
    
}
\lstset{
frame=single, 
breaklines=true,
columns=fullflexible
}
\thiswatermark{\centering \put(0,-105.0){\includegraphics[scale=0.5]{iith.png}} }

\sloppy
\title{\mytitle}
\title{
Probability Assignment-III
}
\author{Nikhil Nair}
\begin{document}
\maketitle
%\tableofcontents
\bigskip


\section{\textbf{Problem }}
Find the mean number of heads in three tosses of a fair coin.


\section{\textbf{Solution }}
Consider each trial results in success (i.e Heads) or failure (i.e Tails).
\\
 
Let p and q = (1 - p) be the probability of success and failure respectively.\\
\begin{align}
p = \frac{1}{2}               \label{1}
\\            
q = 1 - p = \frac{1}{2}       \label{2}
\end{align}\\
In n Bernoulli trials with x success and (n - x) failures, the probablity of x success in n- Bernoulli trials can be given as\\
\begin{align}
^nC_xp^{x} q^{n-x}             \label{3}
\end{align}
\\
Now the distribution of number of successes using $(\ref{1}),(\ref{2}) \& (\ref{3})$ can be given as,
\begin{center}
\begin{tabular}{ |c |c |c |c |c |}
 \hline
 X  &  0 &  1  &  2 & 3\\
 \hline
 P(X)  &  $^3C_0(\frac{1}{2})^{3}$  & $^3C_1(\frac{1}{2})^{3}$  &  $^3C_2(\frac{1}{2})^{3}$ & $^3C_3(\frac{1}{2})^{3}$\\
 \hline
 
\end{tabular}
\end{center}

\begin{align}
\text{Mean of X} = \mu = \sum_{i=1}^{n=3} x_i P(x_i)
\\
\begin{split}
\mu = 0 \times ^3C_0(\frac{1}{2})^{3} + 1 \times ^3C_1(\frac{1}{2})^{3} &+ 2 \times ^3C_2(\frac{1}{2})^{3} \\ &+ 3 \times ^3C_3(\frac{1}{2})^{3}
\end{split}
\\
\therefore \text{Mean of X} = \mu = 1.5
\end{align}

\end{document}
