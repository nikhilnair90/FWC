\documentclass[journal,12pt,twocolumn]{IEEEtran}
\usepackage{graphicx}
\graphicspath{{./figs/}}{}
\usepackage{amsmath,amssymb,amsfonts,amsthm}
\newcommand{\myvec}[1]{\ensuremath{\begin{pmatrix}#1\end{pmatrix}}}
\usepackage{listings}
\usepackage{watermark}
\usepackage{titlesec}
\let\vec\mathbf

\titlespacing{\subsection}{0pt}{\parskip}{-3pt}
\titlespacing{\subsubsection}{0pt}{\parskip}{-\parskip}
\titlespacing{\paragraph}{0pt}{\parskip}{\parskip}
\newcommand{\figuremacro}[5]{
    
}
\lstset{
frame=single, 
breaklines=true,
columns=fullflexible
}
\thiswatermark{\centering \put(0,-105.0){\includegraphics[scale=0.5]{iith.png}} }

\sloppy
\title{\mytitle}
\title{
Probability Assignment-II
}
\author{Nikhil Nair}
\begin{document}
\maketitle
%\tableofcontents
\bigskip



\section{\textbf{Problem }}
Given that the events A and B are such that P(A)=$\frac{1}{2}$, P(A + B)=$\frac{3}{5}$ and P(B)=p. Find p if they are \\
(i)mutually exclusive\\
(ii)independent


\section{\textbf{Solution }}

(i)mutually exclusive
\\
Given A and B are mutually exclusive events,\\
then,
\\

P(A + B) =P(A) + P(B)\\

$\frac{3}{5}=\frac{1}{2}+p$\\

$\therefore  p = \frac{1}{10}$\\

(ii)independent
\\
Given A and B are independent events,\\
then,
\\

P(A + B) =P(A) + P(B) - P(A B)\\


P(A + B) =P(A) + P(B) - P(A)P(B)\\

$\frac{3}{5}=\frac{1}{2}+ p - \frac{p}{2}$\\

$\therefore p = \frac{1}{5}$\\


\end{document}
